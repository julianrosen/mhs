\documentclass[12pt]{amsart}
\usepackage{jrmacros}
%\usepackage[margin=1.2in]{geometry}

\usepackage{mathtools}
\mathtoolsset{showonlyrefs}

%\usepackage{setspace}
%\doublespace

%\numberwithin{equation}{section}

%Theorem envirments
\theoremstyle{plain}
\newtheorem{theorem}{Theorem}%[section]
\newtheorem{proposition}[theorem]{Proposition}
\newtheorem{corollary}[theorem]{Corollary}
\newtheorem{lemma}[theorem]{Lemma}
\newtheorem{conjecture}[theorem]{Conjecture}

\newtheorem*{theorem*}{Theorem}
\newtheorem*{proposition*}{Proposition}
\newtheorem*{corollary*}{Corollary}
\newtheorem*{lemma*}{Lemma}
\newtheorem*{conjecture*}{Conjecture}

\theoremstyle{definition}
\newtheorem{definition}[theorem]{Definition}
\newtheorem{example}[theorem]{Example}
\newtheorem{question}[theorem]{Question}
\newtheorem{philosophy}[theorem]{Philosophy}

\newtheorem*{definition*}{Definition}
\newtheorem*{example*}{Example}
\newtheorem*{question*}{Question}
\newtheorem*{philosophy*}{Philosophy}

\theoremstyle{remark}
\newtheorem{remark}[theorem]{Remark}

\newtheorem*{remark*}{Remark}

\def\A{\hat{\cA}}
\def\A{\mathbb{A}}
\def\z#1{\zeta_{\cA_{#1}}}
\def\I{\mathcal{I}}
\def\ape{Ap\'{e}ry}
\def\li{\mathrm{Li}}
\def\hp{h_{p}}
\def\hn{h^{[n]}}
\def\spa{\mathrm{Span}_{\Q}}
\def\ZZ{\hat{\zeta}}
\def\qsh{\hat{\qs}}
\def\hI{\hat{\I}}
\def\bn{\mathbf{n}}
\def\mz{\cZ_{\cM}}
\def\tmz{\tilde{\mz}}
\def\zm{\zeta^{\cM}}
\def\G{\mathbb{G}}

\def\mm{\cZ'}
\def\MM{\cZ}
\def\zm{\zeta^{\cM}}
\def\Mh{\hat{\MM}}
\def\ph{per_{\A}}
\def\PH{PER_{\A}}
\def\AA{\A[[p]^{-1}]}


\def\p{\mathbf{p}}
\def\A{\cA}
\def\Ah{\hat{\A}}
\def\Ai{\Ah[\p^{-1}]}
\def\M{\mathcal{Z}}
\def\MM{\M'}
\def\per{\hat{per}}
\def\zm{\zeta^{\mathfrak{m}}}
\def\zmm{\zeta^{\MM}}
\def\zb{\overline{\zeta}_p}
\def\PP{\cP}
\def\PPm{\PP^{\cM}}


\def\cl{\mathrm{Cl}}

\def\ai{\mathbb{I}}
\def\hv{\hat{v}}

\def\fil{\mathrm{Fil}}
\def\p{\mathbf{p}}
\def\hm{h^{\cM}}
%\def\spec{\mathrm{Spec}}
\def\gal{\mathrm{Gal}}
\def\am{a^{\cM}}

%\hidenotes

\def\lie{\mathrm{Lie}}
\renewcommand{\thepart}{\Roman{part}}

%Comment macros
\def\todo#1{\inline{Green}{To do:}{#1}}
%\hidenotes

\begin{document}
\title{MHS examples}
\author{Julian Rosen}
\date{\today}
\maketitle
%\begin{abstract}
%\end{abstract}

This document lists some example computations for the Python 2.7 package \verb|mhs|. The software accompanies the papers \cite{Ros16a} and \cite{Ros16b}, and makes use of computations from the MZV data mine \cite{Blu10}. All of the examples in this document can be found in the iPython notebook \verb|mhs_examples.ipynb|. The most recent version can always be found at
\begin{center}
\url{https://sites.google.com/site/julianrosen/mhs}.
\end{center}

\section{Computations}


\noindent First, we import the mhs package. The default weight is $8$.
\bigskip

\begin{verbatim}
>>> from mhs import *
Imported data through weight 9
\end{verbatim}
\bigskip

\noindent The package uses a basis for the space of multiple harmonic sums. Let's see the basis.

\begin{verbatim}

>>> mhs_basis()
[(), (2, 1), (4, 1), (4, 1, 1), (6, 1), (5, 2, 1), (6, 1, 1)]

\end{verbatim}

\noindent In other words, modulo $p^9$, every weighted multiple harmonic sum can we written as a rational linear combination of
\[
1=\hp(\varnothing),\; \hp(2,1),\;\hp(4,1),\;\hp(4,1,1),\;\hp(6,1),\;\hp(5,2,1),\;\hp(6,1,1).
\]
\bigskip

\noindent The computer can compute expansions in terms of this basis.

\begin{verbatim}

>>> a = Hp(1, 2, 3); a.disp()
\end{verbatim}
$\displaystyle\H(1,2,3)$

\begin{verbatim}

>>> a.mhs()
\end{verbatim}
\[
-2H_{p-1}(4, 1, 1)- \frac{11}{28}pH_{p-1}(6, 1)+p^{2}\left(- \frac{27}{7}H_{p-1}(5, 2, 1)-9H_{p-1}(6, 1, 1)\right)+O(p^{3})
\]


\noindent The computer is telling us that the congruence
\begin{gather*}
\H(1,2,3)\equiv -2H_{p-1}(4, 1, 1)- \frac{11}{28}pH_{p-1}(6, 1)\\
\hspace{40mm}+p^{2}\left(- \frac{27}{7}H_{p-1}(5, 2, 1)-9H_{p-1}(6, 1, 1)\right)\mod p^3
\end{gather*}
holds for all sufficiently large $p$, and indeed the computer has found a \emph{proof} of this congruence.
\bigskip

\noindent We can also find expansions in terms of $p$-adic multiple zeta values. For example, binomial coefficients.

\begin{verbatim}

>>> b = binp(5,2); b.disp()
\end{verbatim}
$\displaystyle
{5p \choose 2p}
$
\begin{verbatim}

>>> b.mzv()
\end{verbatim}
\[
10-300p^{3}\zeta_p(3)-5700p^{5}\zeta_p(5)+4500p^{6}\zeta_p(3)^{2}-108300p^{7}\zeta_p(7)+171000p^{8}\zeta_p(3)\zeta_p(5)+O(p^{9})
\]
\smallskip

\noindent This is a lot of output. If we only want an expansion modulo e.g.\  $p^6$:

\begin{verbatim}

>>> b.mzv(err=6)
\end{verbatim}
\[
10-300p^{3}\zeta_p(3)-5700p^{5}\zeta_p(5)+O(p^{6})
\]
\smallskip

\noindent Another quantity we can compute:
\begin{verbatim}

>>> c = curp(2,4); c.disp()
\end{verbatim}
$\displaystyle\sum_{\substack{n_1 + n_2 + n_3 + n_4 = p^2\\p\nmid n_{1}n_{2}n_{3}n_{4}}}\frac{1}{n_1n_2n_3n_4}$

\begin{verbatim}

>>> c.mhs()
\end{verbatim}
$\displaystyle - \frac{24}{5}p^{2}H_{p-1}(4, 1) + \frac{28}{15}p^{3}H_{p-1}(4, 1, 1) + O(p^{4})$
\smallskip

\noindent If we want higher precision, we should import data of higher weight.
\begin{verbatim}
>>> import_data(9)
Imported data through weight 9

>>> c = curp(2,4); c.mhs()
\end{verbatim}
\[
- \frac{24}{5}p^{2}H_{p-1}(4, 1) + \frac{28}{15}p^{3}H_{p-1}(4, 1, 1) + p^{4}\left(- \frac{24}{5}H_{p-1}(4, 1) + \frac{4421}{420}H_{p-1}(6, 1)\right) + O(p^{5})
\]
\smallskip

The software can also handle various other nested sums. For example, we can compute something like
\[
\sum_{n=1}^{p-1}\frac{(-1)^n}{n+2}{p+n+1\choose n-1}{p\choose n}.
\]

\begin{verbatim}
>>> import_data(8)
Imported data through weight 8
>>> a = (BIN(1,-1)*BINN(0,0)*nn(-2)).sum(1,0).e_p()
>>> a.mhs()
\end{verbatim}

\begin{gather*}
\frac{1}{2} -  \frac{5}{4}p + \frac{15}{8}p^{2} -  \frac{41}{16}p^{3} + p^{4}\left(\frac{99}{32} -  \frac{1}{3}H_{p-1}(2, 1)\right) -  \frac{221}{64}p^{5} \\
+ p^{6}\left(\frac{471}{128} + \frac{5}{12}H_{p-1}(2, 1) -  \frac{7}{30}H_{p-1}(4, 1)\right) + O(p^{7})
\end{gather*}

\begin{verbatim}

>>> a.mzv()
\end{verbatim}
\begin{gather*}
\frac{1}{2} -  \frac{5}{4}p + \frac{15}{8}p^{2} -  \frac{41}{16}p^{3} + p^{4}\left(\frac{99}{32} - \zeta_p(3)\right) -  \frac{221}{64}p^{5} \\
+ p^{6}\left(\frac{471}{128} + \frac{5}{4}\zeta_p(3) - 3\zeta_p(5)\right) + O(p^{7})
\end{gather*}
\smallskip

The software uses MathJax to render the output in a Jupyter notebook. If you would prefer to get non-Tex output, this can be done.
\begin{verbatim}

>>> a.mzv(tex=False)
'1/2 -  5/4p + 15/8p^2 -  41/16p^3 + p^4(99/32 - zeta_p(3)) -  221/64p^5
+ p^6(471/128 + 5/4zeta_p(3) - 3zeta_p(5)) + O(p^7)'
\end{verbatim}

\bibliographystyle{hplain}
\bibliography{jrbiblio}
\end{document}
\documentclass[12pt]{amsart}
\usepackage{jrmacros}
%\usepackage[margin=1.2in]{geometry}

\usepackage{mathtools}
\mathtoolsset{showonlyrefs}

%\usepackage{setspace}
%\doublespace

%\numberwithin{equation}{section}

%Theorem envirments
\theoremstyle{plain}
\newtheorem{theorem}{Theorem}%[section]
\newtheorem{proposition}[theorem]{Proposition}
\newtheorem{corollary}[theorem]{Corollary}
\newtheorem{lemma}[theorem]{Lemma}
\newtheorem{conjecture}[theorem]{Conjecture}

\newtheorem*{theorem*}{Theorem}
\newtheorem*{proposition*}{Proposition}
\newtheorem*{corollary*}{Corollary}
\newtheorem*{lemma*}{Lemma}
\newtheorem*{conjecture*}{Conjecture}

\theoremstyle{definition}
\newtheorem{definition}[theorem]{Definition}
\newtheorem{example}[theorem]{Example}
\newtheorem{question}[theorem]{Question}
\newtheorem{philosophy}[theorem]{Philosophy}

\newtheorem*{definition*}{Definition}
\newtheorem*{example*}{Example}
\newtheorem*{question*}{Question}
\newtheorem*{philosophy*}{Philosophy}

\theoremstyle{remark}
\newtheorem{remark}[theorem]{Remark}

\newtheorem*{remark*}{Remark}

%Comment macros
\def\todo#1{\inline{Green}{To do:}{#1}}
%\hidenotes

\begin{document}
\title{Some nice computations the MHS software can do}
\author{Julian Rosen}
\date{\today}
\maketitle
%\begin{abstract}
%\end{abstract}
\[
\sum_{p-1\geq n>m\geq 3}\frac{{p+m\choose m}}{n(m-2)}\equiv -\frac{13}{4}-\lp\frac{31}{4}+4\zeta_p(3)\rp p\mod p^2.
\]
\[
\sum_{p-1\geq n\geq m\geq 3}\frac{{p+m\choose m}}{n(m-2)}\equiv -2-\lp5+4\zeta_p(3)\rp p-12 \zeta_p(3)p^2\mod p^3.
\]
\[
\sum_{n=1}^{p-1}H_n(1,1)\sum_{m=1}^n {m+p+2\choose p}\frac{1}{m^3}\equiv - \frac{1}{8}- \frac{17}{16}p+\left(3+\frac{9}{2}p\right)\zeta_p(3)-11p\,\zeta_p(5)\mod p^2.
\]
\[
\sum_{n=0}^p {p\choose n}^4\equiv 2-16p^{5}\,\zeta_p(5)-20p^{6}\,\zeta_p(3)^{2}-143p^{7}\,\zeta_p(7)+O(p^{8})
\]

%\bibliographystyle{hplain}
%\bibliography{jrbiblio}
\end{document}